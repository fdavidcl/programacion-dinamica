%%%
% Modificación de una plantilla de Latex para adaptarla al castellano.
%%%

%%%%%%%%%%%%%%%%%%%%%%%%%%%%%%%%%%%%%%%%%
% Thin Sectioned Essay
% LaTeX Template
% Version 1.0 (3/8/13)
%
% This template has been downloaded from:
% http://www.LaTeXTemplates.com
%
% Original Author:
% Nicolas Diaz (nsdiaz@uc.cl) with extensive modifications by:
% Vel (vel@latextemplates.com)
%
% License:
% CC BY-NC-SA 3.0 (http://creativecommons.org/licenses/by-nc-sa/3.0/)
%
%%%%%%%%%%%%%%%%%%%%%%%%%%%%%%%%%%%%%%%%%

%----------------------------------------------------------------------------------------
%	PACKAGES AND OTHER DOCUMENT CONFIGURATIONS
%----------------------------------------------------------------------------------------

\documentclass[a4paper, 11pt]{article} % Font size (can be 10pt, 11pt or 12pt) and paper size (remove a4paper for US letter paper)

\usepackage[protrusion=true,expansion=true]{microtype} % Better typography
\usepackage{graphicx} % Required for including pictures
\usepackage{wrapfig} % Allows in-line images
\usepackage[spanish]{babel} % English language/hyphenation
\usepackage[usenames,dvipsnames]{color} % Coloring code
\selectlanguage{spanish}
\usepackage[utf8]{inputenc}

\usepackage{mathpazo} % Use the Palatino font
\usepackage{amsmath}
\usepackage[T1]{fontenc} % Required for accented characters
\setlength{\parindent}{0pt}
\parskip=8pt
\linespread{1.05} % Change line spacing here, Palatino benefits from a slight increase by default

\makeatletter
\renewcommand\@biblabel[1]{\textbf{#1.}} % Change the square brackets for each bibliography item from '[1]' to '1.'
\renewcommand{\@listI}{\itemsep=0pt} % Reduce the space between items in the itemize and enumerate environments and the bibliography

\renewcommand{\maketitle}{ % Customize the title - do not edit title and author name here, see the TITLE block below
\begin{flushright} % Right align
{\LARGE\@title} % Increase the font size of the title

\vspace{50pt} % Some vertical space between the title and author name

{\large\@author} % Author name
\\\@date % Date

\vspace{40pt} % Some vertical space between the author block and abstract
\end{flushright}
}

%----------------------------------------------------------------------------------------
%	TITLE
%----------------------------------------------------------------------------------------

\title{\textbf{Práctica 5}\\ % Title
Programación Dinámica} % Subtitle

\author{\textsc{Óscar Bermúdez,\\Francisco David Charte,\\Ignacio Cordón,\\José Carlos Entrena,\\Mario Román} % Author
\\{\textit{Universidad de Granada}}} % Institution

\date{\today} % Date

%----------------------------------------------------------------------------------------

\begin{document}

\maketitle % Print the title section


%----------------------------------------------------------------------------------------
%	ABSTRACT AND KEYWORDS
%----------------------------------------------------------------------------------------

\renewcommand{\abstractname}{Resumen} % Uncomment to change the name of the abstract to something else

%\begin{abstract}
%Morbi tempor congue porta. Proin semper, leo vitae faucibus dictum, metus mauris lacinia lorem, ac congue leo felis eu turpis. Sed nec nunc pellentesque, gravida eros at, porttitor ipsum. Praesent consequat urna a lacus lobortis ultrices eget ac metus. In tempus hendrerit rhoncus. Mauris dignissim turpis id sollicitudin lacinia. Praesent libero tellus, fringilla nec ullamcorper at, ultrices id nulla. Phasellus placerat a tellus a malesuada.
%\end{abstract}

%\hspace*{3,6mm}\textit{Keywords:} lorem , ipsum , dolor , sit amet , lectus % Keywords

%\vspace{30pt} % Some vertical space between the abstract and first section


{\parskip=2pt
  \tableofcontents
}
\pagebreak


%% Estructura de la práctica
% Para cada algoritmo, debería escribirse:
%  - Enunciado formal del problema
%  - Función que se memoriza para aplicar programación dinámica.
%  - Implementación en Ruby o pseudocódigo
%  - Análisis de eficiencia
%  - Implementación en C++ si es necesaria por eficiencia

\section{Compraventa de acciones}
	\subsection{Enunciado}
      Dado un array con $n$ valores numéricos, el objetivo es calcular las posiciones $a$ y $b$ que nos den la mayor diferencia $valores[b]-valores[a]$, con la condición de que la posición $b$ esté más a la derecha que $a$.
      
      El problema representa una operación de compra y venta de acciones, en la que se quiere obtener la mayor ganancia. En este caso, el array indica los precios de las acciones a lo largo de un período de tiempo.
      
	\subsection{Algoritmo}
		Para aplicar programación dinámica al problema, hemos de caracterizar una solución óptima. Esta será un par $(a,b)$ cumpliendo
		$$ valores[b]-valores[a] \le valores[j]-valores[i]
		\quad \forall i,j \in \{1,\dots n\}\ con\ i<j $$
		
		Podemos observar que se cumple el principio de optimalidad de Bellman: el par $(a,b)$, solución óptima de los valores de 1 hasta $n$, estará compuesta de la solución $(a',b')$ dada por el mismo problema en los valores de 1 a $n-1$, y la comprobación de si la diferencia del mínimo de $valores$ hasta $n-1$ con $valores[n]$ es mayor que el beneficio de la solución $(a', b')$.
		
		Pseudocódigo del algoritmo:
		
		\small
		\texttt{% Generator: GNU source-highlight, by Lorenzo Bettini, http://www.gnu.org/software/src-highlite
\noindent
\mbox{}\textbf{\textcolor{Blue}{def}}\ resuelve\textcolor{BrickRed}{(}precios\textcolor{BrickRed}{)} \\
\mbox{}\ \ \ \ minimo\ \textcolor{BrickRed}{=}\ \textcolor{Purple}{0} \\
\mbox{}\ \ \ \ compra\ \textcolor{BrickRed}{=}\ \textcolor{Purple}{0} \\
\mbox{}\ \ \ \ venta\ \textcolor{BrickRed}{=}\ \textcolor{Purple}{0} \\
\mbox{} \\
\mbox{}\ \ \ \ \textbf{\textcolor{Blue}{for}}\ i\ \textbf{\textcolor{Blue}{in}}\ \textcolor{BrickRed}{[}\textcolor{Purple}{1}\textcolor{BrickRed}{..}\#precios\textcolor{BrickRed}{]} \\
\mbox{}\ \ \ \ \ \ \ \ \textbf{\textcolor{Blue}{if}}\ \textcolor{BrickRed}{(}precios\textcolor{BrickRed}{[}i\textcolor{BrickRed}{]}\ \textcolor{BrickRed}{\textless{}}\ precios\textcolor{BrickRed}{[}minimo\textcolor{BrickRed}{]}\textcolor{BrickRed}{)} \\
\mbox{}\ \ \ \ \ \ \ \ \ \ \ \ minimo\ \textcolor{BrickRed}{=}\ i \\
\mbox{} \\
\mbox{}\ \ \ \ \ \ \ \ beneficio\ \textcolor{BrickRed}{=}\ precios\textcolor{BrickRed}{[}i\textcolor{BrickRed}{]}\ \textcolor{BrickRed}{-}\ precios\textcolor{BrickRed}{[}minimo\textcolor{BrickRed}{]} \\
\mbox{} \\
\mbox{}\ \ \ \ \ \ \ \ \textbf{\textcolor{Blue}{if}}\ \textcolor{BrickRed}{(}beneficio\ \textcolor{BrickRed}{\textgreater{}}\ precios\textcolor{BrickRed}{[}venta\textcolor{BrickRed}{]}\ \textcolor{BrickRed}{-}\ precios\textcolor{BrickRed}{[}compra\textcolor{BrickRed}{])} \\
\mbox{}\ \ \ \ \ \ \ \ \ \ \ \ compra\ \textcolor{BrickRed}{=}\ minimo \\
\mbox{}\ \ \ \ \ \ \ \ \ \ \ \ venta\ \textcolor{BrickRed}{=}\ i \\
\mbox{} \\
\mbox{}\ \ \ \ \textbf{\textcolor{Blue}{return}}\ \textcolor{BrickRed}{(}compra\textcolor{BrickRed}{,}\ venta\textcolor{BrickRed}{)} \\
\mbox{}\textbf{\textcolor{Blue}{end}} \\
\mbox{}
}
		\normalsize
		
	\subsection{Análisis de eficiencia}
		El análisis de este algoritmo es muy sencillo. Tenemos un único bucle de tipo \textit{for} realizando $n-1$ iteraciones, con operaciones $\mathcal{O}(1)$ en el interior. Por tanto, su eficiencia será $\mathcal{O}(n)$, es decir, es un algoritmo lineal.

\section{Terminales de venta}
  \subsection{Enunciado}
    Dado un conjunto de monedas $M = \{m_1, m_2, \dots, m_n\} \subset \mathbb{N}$, buscamos frecuencias $\{f_1,f_2,\dots f_n\} \subset \mathbb{N}$ de 
    monedas que cumplan que su suma es igual a un precio dado:
    \begin{equation}
     \sum_{i=1}^n m_i f_i = P
    \end{equation}
    Y que minimicen el número total de monedas:
    \begin{equation}
     \sum_{i=1}^n f_i
    \end{equation}
    
    En nuestro caso, vamos a suponer además que las monedas están ordenadas como
    $m_1 < m_2 < \dots < m_n$. 
    
  \subsection{Algoritmo}
    Podemos aplicar programación dinámica sabiendo que se cumple el principio de optimalidad.
    Cualquier subconjunto de una solución óptima debería ser también una solución óptima para
    su suma, ya que si no lo fuera, podría sustituirse por el óptimo, mejorando el óptimo global.
    
    En particular, el subconjunto de monedas sin la última moneda es también una solución óptima.
    Definimos la función $g: \mathbb{N} \to \mathbb{N}$, donde $g(P)$ será el número
    mínimo de monedas necesarias para sumar un precio $P$. Conocemos el caso base trivial $g(0) = 0$ y
    usando la propiedad anterior, definimos la siguiente relación de recurrencia:
    \begin{equation}
      g(p) = \min_{m_i \in M} (1 + g(p - m_i))
    \end{equation}
    Ya que la solución óptima debe ser de esa forma para alguna moneda $m_i$.
  
    Aplicamos programación dinámica memorizando las imágenes por esta función.
    
    Pseudocódigo del algoritmo:
    
 		\small
 		\texttt{% Generator: GNU source-highlight, by Lorenzo Bettini, http://www.gnu.org/software/src-highlite
\noindent
\mbox{}\textbf{\textcolor{Blue}{def}}\ num$\_$coins\textcolor{BrickRed}{(}tipos\textcolor{BrickRed}{,}\ precio\textcolor{BrickRed}{)} \\
\mbox{}\ \ \ \ monedas\ \textcolor{BrickRed}{=}\ \textcolor{BrickRed}{[]} \\
\mbox{}\ \ \ \ monedas\textcolor{BrickRed}{[}\textcolor{Purple}{0}\textcolor{BrickRed}{]}\ \textcolor{BrickRed}{=}\ \textcolor{Purple}{0} \\
\mbox{} \\
\mbox{}\ \ \ \ \textbf{\textcolor{Blue}{for}}\ c\ \textbf{\textcolor{Blue}{in}}\ \textcolor{BrickRed}{[}\textcolor{Purple}{0}\textcolor{BrickRed}{..}precio\textcolor{BrickRed}{]} \\
\mbox{}\ \ \ \ \ \ \ \ \textbf{\textcolor{Blue}{for}}\ m\ \textbf{\textcolor{Blue}{in}}\ tipos \\
\mbox{}\ \ \ \ \ \ \ \ \ \ \ \ \textbf{\textcolor{Blue}{if}}\ \textcolor{BrickRed}{(}c\textcolor{BrickRed}{+}m\ \textcolor{BrickRed}{$\le$}\ precio\ \textcolor{BrickRed}{\&\&} \\
\mbox{}\ \ \ \ \ \ \ \ \ \ \ \ \ \ \ \ \textcolor{BrickRed}{(!}monedas\textcolor{BrickRed}{[}c\textcolor{BrickRed}{+}m\textcolor{BrickRed}{]}\ \textcolor{BrickRed}{$|$$|$}\ monedas\textcolor{BrickRed}{[}c\textcolor{BrickRed}{]}\ \textcolor{BrickRed}{+}\ \textcolor{Purple}{1}\ \textcolor{BrickRed}{\textless{}}\ monedas\textcolor{BrickRed}{[}c\textcolor{BrickRed}{+}m\textcolor{BrickRed}{])} \\
\mbox{}\ \ \ \ \ \ \ \ \ \ \ \ \ \ \ \ monedas\textcolor{BrickRed}{[}c\textcolor{BrickRed}{+}m\textcolor{BrickRed}{]}\ \textcolor{BrickRed}{=}\ monedas\textcolor{BrickRed}{[}c\textcolor{BrickRed}{]}\ \textcolor{BrickRed}{+}\ \textcolor{Purple}{1} \\
\mbox{} \\
\mbox{}\ \ \ \ \textbf{\textcolor{Blue}{return}}\ monedas\textcolor{BrickRed}{[}precio\textcolor{BrickRed}{]} \\
\mbox{}\textbf{\textcolor{Blue}{end}} \\
\mbox{}
}
 		\normalsize
 	
 	\subsection{Análisis de eficiencia}
 	En este algoritmo, utilizamos dos bucles anidados,
 	el exterior realiza \texttt{precio} iteraciones, y el interior recorre los tipos, luego hará tantas iteraciones como tipos haya. En resumen, la eficiencia queda $\mathcal{O}(np)$, donde $n$ es la cantidad de tipos distintos de monedas y $p$ es el precio a resolver.
  
\section{Problema del viajante de comercio}
  \subsection{Enunciado}
  	Dada una lista $S$ de $n$ ciudades, representadas como puntos en el plano:
  	\begin{equation}
  	    S = [(x_0,y_0), (x_1,y_1), \dots (x_{n-1},y_{n-1})] \subset \mathbb{R}^2
  	\end{equation}
  	Y definiendo la longitud de recorrer una lista como la suma de las distancias de cada ciudad a la siguiente:
  	\begin{equation}
  	    long(S) = \sum_{i \in \mathbb{Z}_n} dist((x_i,y_i), (x_{i+1}, y_{i+1})) = \sum_{i \in \mathbb{Z}_n} \sqrt{(x_i-x_{i+1})^2 + (y_i-y_{i+1})^2}
  	\end{equation}
  	El objetivo es encontrar la permutación de la lista $\sigma : \mathbb{Z}_n \leftrightarrow \mathbb{Z}_n$, verificando que su longitud sea mínima:
  	\begin{equation}
  	    long(\sigma(S)) = long([(x_{\sigma(1)},y_{\sigma(1)}), (x_{\sigma(2)},y_{\sigma(2)}), \dots, (x_{\sigma(n)},y_{\sigma(n)})])
      \end{equation}
      

\section{Caminos mínimos en grafos}
  \subsection{Enunciado}
    %Traducido de: http://www.cs.rochester.edu/u/nelson/courses/csc_173/graphs/apsp.html, entre otras
    Sea $G=(V,E)$ un grafo con $n$ vértices $V$ y un conjunto de aristas $E$,
    donde cada arista $(u,v)$ tiene asociado un peso $w(u,v)$.
    
    %http://es.wikipedia.org/wiki/Camino_%28teor%C3%ADa_de_grafos%29
    Definimos camino como una secuencia de vértices del grafo de manera que exista una arista entre cada vértice y el siguiente. Se dice que es simple si no repite vértices. Si el camino empieza y termina en el mismo nodo, se llama ciclo.
    
    Definimos el coste asociado a un camino como la suma de los pesos de las aristas usadas para unir los vértices de dicho camino.
     
    Llamamos camino de coste mínimo al camino con nodo origen $s$ y destino $a$ de menor coste asociado.
    
    El problema del \textit{All Pairs Shortest Path} (APSP) consiste en encontrar el camino de coste mínimo para cualquier pareja de puntos $u$ y $v$.
    
  \subsection{Algoritmo de Dijkstra}
    Para el algoritmo de Dijkstra asumimos que el grafo no contiene aristas de peso negativo.
    Fijado un vértice $s$, definimos $g(a)$ como la distancia del camino mínimo entre $s$ y $a$.
    
    Para calcular $g(a)$ para todos los puntos:
    \begin{itemize}
      \item Al nodo fijado $s$, le asignamos el valor 0 y lo tomamos como referente $r$ y al resto le asignamos $\infty$
      \item A todos los puntos que estén unidos al nodo $r$  por una
      arista y no hayan sido tomados como referente antes, $a$, le asignamos $g(a) = \min \{g(a), g(r) + w(r,a)\}$.
      \item Tomamos como referente $r$ el siguiente nodo del grafo con el menor valor.
      \item Repetimos los puntos 2 y 3 hasta que todos los nodos hayan sido tomados como referente.
    \end{itemize}
    
    Para el problema APSP se nos pide que apliquemos este algoritmo $n$ veces sobre cada uno
    de los posibles nodos iniciales. La función recursiva queda definida como:
    \begin{equation} 
      \begin{split}
	g(s)  &=  0 \\
	g(a)  &=  \min_{r \in V} \{g(r) + w(r,a)\} \\
      \end{split}
    \end{equation}
    Nótese que si $a,r$ no están conectados, el peso de su arista es infinito.
  
    \subsubsection{Requisitos y eficiencia}
      Para poder ejecutar el algoritmo de Dijkstra debemos requerir que el grafo
      no contenga aristas de peso negativo.
      
      En un análisis de la eficiencia del algoritmo de Dijkstra, comprobamos que se realizan $n-1$ actualizaciones de la distancia, una por cada nodo que tomamos como referente. Para calcular esa distancia, realizamos un máximo de $n-1$ comparaciones, seguidas de una suma y una comparación para actualizar la etiqueta de cada nodo, lo que nos dice que el algoritmo tendrá un orden de eficiencia cuadrático. Como lo empleamos en cada nodo del grafo, el algoritmo final tendrá una eficiencia $O(n^3)$. 

  \subsection{Algoritmo de Bellman-Ford}
    Para el algoritmo de Bellman-Ford asumimos que el grafo no contiene ciclos de peso negativo.
    Fijado un vértice $s$, definimos $g(a,m)$ como la distancia del camino mínimo entre $s$ y $a$
    usando hasta $m$ aristas.
    
    Trivialmente, la distancia mínima desde $s$ hasta $a$ será $g(a,n-1)$, ya que no puede
    haber ciclos de peso negativo. Podemos calcular entonces la función usando recursividad:
    \begin{equation} 
      \begin{split}
	g(s,0)  &=  0 \\
	g(u,0)  &=  \infty \\
	g(u,m)  &=  \min_{(u,v) \in E} \{g(v,m-1) + w(v,u)\} \\
      \end{split}
    \end{equation}

    Como en el caso anterior, para el problema APSP se aplicará este algoritmo $n$ veces sobre cada uno
    de los posibles nodos iniciales.

    \subsubsection{Requisitos y eficiencia}
      Para poder ejecutar el algoritmo de Bellman-Ford debemos requerir que el grafo
      no contenga ciclos de peso negativo.
      
      La eficiencia del algoritmo se puede extraer de los cálculos que se realizan para la función $g$. Para un nodo origen dado, el algoritmo realiza tantos cálculos como vértices haya, para cada $m = 1,\dots n-1$. Por tanto, el algoritmo de Bellman-Ford tiene una eficiencia de $\mathcal{O}(|V|\times n)$ con $|V|$ el número de vértices.
      
      Como consecuencia, el algoritmo que utiliza Bellman-Ford para extraer caminos mínimos para cada vértice tiene una eficiencia de $\mathcal{O}(|V|^2\times n)$.
      
      
      
    \subsection{Algoritmo de Floyd-Warshall}
      Para el algoritmo de Floyd-Warshall asumimos que el grafo no contiene ciclos de peso negativo.
      Dados vértices $u,v \in V$, llamaremos $g(u,v,m)$ a la distancia del camino más corto de $u$
      hasta $v$ pudiendo usar sólo los nodos $[1..m]$ como nodos intermedios.
      
      Podemos calcular entonces la función usando recursividad, sabiendo que el
      camino mínimo está obligado a pasar o no pasar por el nodo $k$ y que los
      subcaminos de un camino mínimo son también mínimos:
      \begin{equation}
	\begin{split}
	g(u,v,0) &= w(u,v) \\
	g(u,v,m) &= min\{g(u,v,m-1), g(u,k,k-1)+g(k,v,k-1)\}
	\end{split}
      \end{equation}
    
    \subsubsection{Requisitos y eficiencia}
      Para poder ejecutar el algoritmo de Floyd-Warshall debemos requerir que el grafo
      no contenga ciclos de peso negativo.
      
      El algoritmo tendrá una eficiencia de $O(n^3)$, ya que necesitaremos calcular
      todos los valores de la función ternaria $g$, y cada uno de ellos se calcula
      en tiempo $O(1)$ suponiendo calculados los que dependen de él, que serán valores
      memoizados.
  
  \subsection{Algoritmo de Johnson}
      Exigimos que el grafo sea dirigido disperso (número de aristas muy inferior al del
      máximo posible en el grafo) y sin ciclos negativos.
      
      %% Notación del enunciado
      %Sea un grafo $G=(V,E)$ con pesos $w(i,j)$. 
      
    \begin{itemize}
      \item Añadimos un vértice $q$ al grafo, definiendo $w(q,u) = 0 \quad \forall u \in V$
      \item Se emplea el algoritmo de Bellman-Ford partiendo de $q$, definiendo
	$g(v)$ como el peso del camino mínimo de $q$ a $v$
      \item Se redefine $w(u,v):=w(u,v) + g(u)-g(v)$. Dado un camino de origen $u$ y
	extremo $v$ que pasa por nodos intermedios $[i_1,\ldots i_p]$ su coste es:
	
	\begin{eqnarray*}
	w(u,i_1) + g(u) -g(i_1) + \sum_{k=1}^{p-1} [w(i_{k},i_{k+1}) g(i_k) - g(i_{k+1}) + w(i_p,v) + \\
	+ g(i_p)- g(v) = w(u,i_1) + \sum_{k=1}^{p-1} [w(i_{k},i_{k+1}) + w(i_p,v) + g(u) - g(v)
	\end{eqnarray*}
	
	Por tanto a todos los caminos que van de $u$ a $v$ en el grafo de nuevos pesos
	se le suma la constante $g(u) - g(v)$.
      \item Usamos el algoritmo de Dijkstra en el grafo de nuevos pesos.
    \end{itemize}
    
      \subsubsection{Requisitos y eficiencia}
	Para poder ejecutar el algoritmo de Johnson, dado que utiliza el algoritmo de Bellman-Ford tenemos los mismos 
	requisitos, es decir, debemos requerir que el grafo no contenga ciclos de peso negativo.
	
\end{document}
